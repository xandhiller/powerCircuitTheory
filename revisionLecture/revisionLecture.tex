\documentclass{article}
\author{Alex Hiller}
\title{Revision Lecture Notes}

% Type-setting
\setlength{\parindent}{0cm}
\setlength{\parskip}{0.25cm}
\pagenumbering{gobble}
\usepackage[margin=2cm]{geometry} % Formatting

% Packages
\usepackage{biblatex}  % Referencing
\addbibresource{/home/polluticorn/GitHub/configuration/list.bib}
\usepackage{amsmath}                  % Mathematics
\usepackage{amssymb}                  % Mathematics
\usepackage{listings}                 % Listings
\usepackage{color}                    % Listings
\usepackage{courier}                  % Listings
\usepackage{circuitikz}               % Circuits
\usepackage{titlesec}                 % Section Formatting

% Code Formatting 
\input{/home/polluticorn/GitHub/texTemplates/codeFormat}

% Math Macros
\input{/home/polluticorn/GitHub/texTemplates/mathMacros}  

% Section formatting
\titleformat{\section}{\huge \bfseries}{}{0em}{}[]
\titleformat{\subsection}{\Large \bfseries}{}{0em}{}
\titleformat{\subsubsection}{\bfseries}{}{0em}{}

 % Note Taking Macros
\input{/home/polluticorn/GitHub/texTemplates/noteTaking}

\begin{document}
% \AtEndDocument{\printbibliography}
\AtEndDocument{\textbf{Questions:} \vspace{3mm} \newline}
\maketitle

We had ten topics.

Three phase circuits -- we knew how to get the $\pi$ model.

We learnt about the faults we have on those systems, what they look like. Including whether they were symmetrical or unsymmetrical.

If we have a $\pi$ model we know how to do analysis to find transients. 

From the transient we can also see how assess find the stability. 

Hints/Comments:
\begin{itemize}
  \item 2hrs + 10mins    
  \item 4 questions
  \item Open book
  \item Non-programmable calculator
  \item People can spend too long searching through their notes, try to avoid this.
  \item Hint: All the basic material you need is on the power-point slides.
  \item Large Laplace transform tables are a distraction. Try to avoid this.
  \item All matrices will either be totally imaginary or totally real. There will be no matrices with complex components. 
  \item If you cannot solve the question numerically, then \textit{nicely solve} the question symbolically and you could get up to 70\% of the marks.
\end{itemize}

Questions:
\begin{itemize}
  \item Question One: AC single phase circuit analysis.
  \item Question Two: Transmission Line modelling. LC(R) Model $\leftarrow$ $VR \%, \ n$ (efficiency) 
  \item Question Three: Fault Analysis. Symmetrical Fault. Can use Thevenin or Z-bus. Ha Pham recommends the Z-bus approach. There will be less than or equal to 4 buses $\Rightarrow$ $4 \times 4$ matrix. 
  \item Question Four: Unsymmetrical Fault, and/or Stability, and/or Transient.
\end{itemize}

\qanda{Topics to study for the final exam?}{As listed above.}
\qanda{Last topic -- generators -- going to be in the exam?}{No.}
\qanda{Long line model -- complicated. Would it be in the final exam?}{Could be. Ha Pham didn't want to specify which it would be. Likely to be Medium Line or Long Line.}


\end{document}
