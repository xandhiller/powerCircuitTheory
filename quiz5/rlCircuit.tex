\documentclass{article}
\author{Alex Hiller}
\title{Embedded Deformation of 3D Heart Image}
\setlength{\parindent}{0cm}
\setlength{\parskip}{0.125cm}

% Packages
\usepackage{amsmath}              % Mathematics
\usepackage{amssymb}
\usepackage[margin=3cm]{geometry} % Formatting
\usepackage{listings}             % Listings
\usepackage{color}                % Listings
\usepackage{courier}              % Listings

% Listing Pre-Requisites
\definecolor{codegreen}{rgb}{0,0.6,0}
\definecolor{codegray}{rgb}{0.5,0.5,0.5}
\definecolor{codepurple}{rgb}{0.58,0,0.82}
\definecolor{backcolour}{rgb}{0.95,0.95,0.92}
\lstdefinestyle{mystyle}{
  backgroundcolor=\color{backcolour},   
  commentstyle=\color{codegreen},
  keywordstyle=\color{magenta},
  numberstyle=\tiny\color{codegray},
  stringstyle=\color{codepurple},
  basicstyle=\footnotesize,
  breakatwhitespace=false,         
  breaklines=true,                 
  captionpos=b,                    
  keepspaces=true,                 
  numbers=left,                    
  numbersep=5pt,                  
  showspaces=false,                
  showstringspaces=false,
  showtabs=false,                  
  tabsize=2
}
\lstset{style=mystyle} 

% Laplace transform macro
\newcommand{\Lapl}[1]{\mathcal{L} \ \bigg\{ #1  \bigg\}  } 

\begin{document}

\textbf{Question:} \\
When $t=0$ a switch closes connecting a voltage source to a series RL circuit. The source is characterised by: $v(t) = V_m \sin(\omega t + \phi)$. Find the time domain expression of the current, $i(t)$.

\begin{equation}
  I(s) = \frac{V(s)}{Z(s)}
\end{equation}

Finding $V(s)$ 

\begin{equation}
  \Lapl{v(t)} =
  \Lapl{ V_m \sin(\omega t + \phi) } \ =
  V_m \Lapl{ \sin(\omega t + \phi) } 
\end{equation}

\begin{equation}
  V_m \ \Lapl{\sin(\omega t)\cos(\phi) +
  \cos(wt)\sin(\phi)}
\end{equation}

\begin{equation}
  V_m  \bigg( \ \Lapl{\sin(\omega t)\cos(\phi)} +
  \Lapl{\cos(wt)\sin(\phi)} \bigg)
\end{equation}

Note that $\cos(\phi)$ and $\sin(\phi)$ are constants with respect to time can subsequently be removed from the argument of the Laplace transform.

\begin{equation}
   V_m  \bigg( \cos(\phi) \ \Lapl{\sin(\omega t)} +
   \sin(\phi) \ \Lapl{\cos(wt)} \bigg)
\end{equation}

\begin{equation}
  \therefore \ V(s) = 
  V_m  \bigg( \cos(\phi) \ \frac{\omega}{s^2 +
  \omega^2}  +
  \sin(\phi) \ \frac{s}{s^2+\omega^2} \bigg)
\end{equation}

Finding Z(s) by simple s-domain series impedance:

\begin{equation}
  Z(s) =
  sL + R \quad \Rightarrow \quad \frac{1}{Z(s)} =
  \frac{1}{sL + R} =
  \frac{\frac{1}{L}}{s+\frac{R}{L}}
\end{equation}

Then to current, $I(s)$ 

\begin{equation}
  I(s) = 
  \frac{V(s)}{Z(s)} =  
  V_m  \bigg( \cos(\phi) \ \frac{\omega}{s^2 + 
  \omega^2}  + \sin(\phi) \ \frac{s}{s^2+\omega^2} \bigg) \frac{\frac{1}{L}}{s+\frac{R}{L}}
\end{equation}

\begin{equation}
   =  V_m  \bigg(  \frac{\cos(\phi) \omega + 
   \sin(\phi) \ s}{s^2 + \omega^2} \bigg) \frac{\frac{1}{L}}{s+\frac{R}{L}}
\end{equation}

\begin{equation}
  \frac{V_m}{L}<++>
\end{equation}



<++>

\end{document}
